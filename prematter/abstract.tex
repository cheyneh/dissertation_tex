% Write in only the text of your abstract, all the extra heading jargon is automatically taken care of
\begin{abstract}

  This dissertation presents a multifaceted look into the structural
  decomposition of permutation classes. The theory of permutation patterns is a
  rich and varied field, and is a prime example of how an accessible and
  intuitive definition leads to increasingly deep and significant line of
  research. The use of geometric structural reasoning, coupled with analytic
  and probabilistic techniques, provides a concrete framework from which to
  develop new enumerative techniques and forms the underlying foundation to
  this study. 

  This work is divided into five chapters. The first chapter introduces these
  techniques through working examples, both motivating the use of structural
  decomposition and showcasing the utility of their combination with
  analytic and probabilistic methods. The remaining chapters apply these
  concepts to separate aspects of permutation classes, deriving new
  enumerative, statistical, and structural results. These chapters are largely
  independent, but build from the same foundation to construct an overarching
  theme of building structure upon disorder.

  The main results of this study are as follows. Chapter~\ref{chap:expat}
  investigates the average number of occurrences of patterns with permutation
  classes, and proves that the total number of 231-patterns is the same in the
  classes of 132- and 123-avoiding permutations. Chapter~\ref{chap:involutions}
  applies structural decomposition to enumerate pattern avoiding involutions.
  Chapter~\ref{chap:polyclass} uses the theory of grid classes to develop an
  algorithm to enumerate the so-called polynomial permutation classes, and
  applies this to the biological problem of genetic evolutionary distance.
  Finally, we end in Chapter~\ref{chap:fixpat} with an exploration of
  pattern-packing, and determine the probability distribution for the number
  of distinct large patterns contained in a permutation. 

    
\end{abstract}
